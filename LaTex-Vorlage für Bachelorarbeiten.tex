\documentclass[12pt, a4paper, oneside]{article}

% Pakete für Formatierung und Schrift
\usepackage{geometry}
\usepackage{amsmath}
\usepackage{graphicx}
\usepackage{acronym}
\usepackage{listings}
\usepackage{setspace}
\usepackage{hyperref}
\usepackage{tabularx}
\usepackage{longtable}
\usepackage{mathptmx}  % Times Schriftart für PDFLaTeX

% Seitengeometrie gemäß Vorgaben
\geometry{top=4cm, bottom=2cm, left=4cm, right=2cm}

% Zeilenabstand und Absatzabstand
\onehalfspacing
\setlength{\parskip}{6pt} % Abstand nach Absätzen
\setlength{\parindent}{0cm} % Keine Einrückung bei neuen Absätzen

% Einstellungen für Überschriften
\usepackage{titlesec}
\titleformat{\section}{\bfseries\fontsize{16pt}{16pt}\selectfont}{\thesection}{1em}{}
\titleformat{\subsection}{\bfseries\fontsize{14pt}{14pt}\selectfont}{\thesubsection}{1em}{}
\titleformat{\subsubsection}{\bfseries\fontsize{12pt}{12pt}\selectfont}{\thesubsubsection}{1em}{}

\titlespacing*{\section}{0pt}{12pt}{6pt}
\titlespacing*{\subsection}{0pt}{12pt}{6pt}
\titlespacing*{\subsubsection}{0pt}{12pt}{6pt}

% Fußnoten
\renewcommand{\footnotesize}{\fontsize{10pt}{12pt}\selectfont}

% Einstellungen für den Quellcode
\lstset{
	language=R,
	basicstyle=\ttfamily\small,
	literate={\u00d6}{{\"O}}1 {\u00c4}{{\"A}}1 {\u00dc}{{\"U}}1 {\u00df}{{\ss}}1 {\u00fc}{{\"u}}1 {\u00e4}{{\"a}}1 {\u00f6}{{\"o}}1
}

\begin{document}
	
	% Deckblatt
	\begin{titlepage}
		\centering
		% \includegraphics[width=0.3\textwidth]{logo.png} % Logo falls gewünscht hier einfügen
		
		\textbf{Anonyme Hochschule für Ghostwriting}\\
		\textbf{Studienzentrum Beispielstadt}\vspace{2cm}
		
		\textbf{\LARGE Bachelorarbeit}\vspace{1.5cm}
		
		zur Erlangung des Grades eines\\
		\textbf{Bachelor of Science (B. Sc.)}\vspace{1.5cm}
		
		über das Thema\\
		\textbf{\LARGE Beispielthema im Beispielkontext}\\
		\textbf{\LARGE – eine beispielhafte Studie}\vspace{1.5cm}
		
		von\\
		Max Beispielstudent\vspace{2cm}
		
		\begin{flushleft}
			Erstgutachter: \hspace{0.3cm} Prof. Dr. Beispiel Dozent\\
			Matrikelnummer: \hspace{0.3cm} 123456\\
			Abgabedatum: \hspace{0.3cm} TT.MM.JJJJ
		\end{flushleft}
	\end{titlepage}
	\newpage  % Seitenumbruch nach dem Deckblatt
	
	% Inhaltsverzeichnis
	\begin{spacing}{1}
		\tableofcontents
		\newpage
	\end{spacing}
	
	% Abbildungsverzeichnis
	\section*{Abbildungsverzeichnis}
	\addcontentsline{toc}{section}{Abbildungsverzeichnis}
	\listoffigures
	\newpage
	
	% Tabellenverzeichnis
	\section*{Tabellenverzeichnis}
	\addcontentsline{toc}{section}{Tabellenverzeichnis}
	\listoftables
	\newpage
	
	% Abkürzungsverzeichnis
	\section*{Abkürzungsverzeichnis}
	\addcontentsline{toc}{section}{Abkürzungsverzeichnis}
	\begin{acronym}
		\acro{BI}{Business Intelligence}
		% Weitere Abkürzungen hier hinzufügen
	\end{acronym}
	\newpage
	
	% Sperrvermerk (optional)
	\section*{Sperrvermerk}
	\addcontentsline{toc}{section}{Sperrvermerk}
	Die vorliegende Arbeit enthält vertrauliche Informationen und ist nur für die Hochschule sowie die Gutachter bestimmt.
	\newpage
	
	% Einleitung
	\section{Einleitung}
	\subsection{Problemstellung der Arbeit}
	Lorem ipsum dolor sit amet, consectetur adipisici elit, sed eiusmod tempor incidunt ut labore et dolore magna aliqua.
	
	\subsection{Zielsetzung der Arbeit}
	Ut enim ad minim veniam, quis nostrud exercitation ullamco laboris nisi ut aliquip ex ea commodo consequat.
	
	\subsection{Aufbau der Arbeit}
	Duis aute irure dolor in reprehenderit in voluptate velit esse cillum dolore eu fugiat nulla pariatur.
	
	% Methodisches Vorgehen
	\section{Methodisches Vorgehen}
	\subsection{Literaturrecherche}
	Beschreiben Sie die Methodik der Literaturrecherche.
	
	\subsection{Qualitative Forschung: Experteninterviews}
	Erläutern Sie die qualitative Forschungsmethode.
	
	\subsection{Prototyping}
	Diskutieren Sie den Prototyping-Ansatz.
	
	\subsection{Quantitative Forschung: Fragebogen}
	Beschreiben Sie die quantitative Forschungsmethodik mittels Fragebogen.
	
	% Konzeptioneller Hintergrund
	\section{Konzeptioneller Hintergrund}
	Einführung in die theoretischen Grundlagen, die den Rahmen der Arbeit bilden.
	
	% Ergebnisse
	\section{Ergebnisse}
	Vorstellung der Forschungsergebnisse.
	
	% Schluss
	\section{Schluss}
	\subsection{Zusammenfassung}
	Zusammenfassung der wichtigsten Ergebnisse.
	
	\subsection{Ausblick}
	Ein kurzer Ausblick auf mögliche zukünftige Forschungen.
	
	% Anhang
	\newpage
	\section*{Anhang}
	\addcontentsline{toc}{section}{Anhang}
	Zusätzliche Informationen und Dokumente, die im Hauptteil der Arbeit nicht untergebracht wurden.
	
	% Literaturverzeichnis
	\newpage
	\section{Literatur- und Internetquellenverzeichnis}
	\subsection{Literaturverzeichnis}
	% \bibliographystyle{gerplain}
	% \bibliography{literatur}
	
	\subsection{Internetquellenverzeichnis}
	Internetquellen separat auflisten.
	
	% Eidesstaatliche Erklärung
	\newpage
	\section*{Eidesstaatliche Erklärung}
	\addcontentsline{toc}{section}{Eidesstaatliche Erklärung}
	Hiermit versichere ich, dass ich die vorliegende Arbeit selbstständig und ohne unerlaubte Hilfe angefertigt habe.
	
\end{document}
